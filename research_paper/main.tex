%&latex
\documentclass[12pt]{article}
\usepackage{amsmath}
\usepackage{graphicx,psfrag,epsf}
\usepackage{enumerate}
\usepackage{natbib}
\usepackage{url} % not crucial - just used below for the URL 

%\pdfminorversion=4
% NOTE: To produce blinded version, replace "0" with "1" below.
\newcommand{\blind}{0}

% DON'T change margins - should be 1 inch all around.
\addtolength{\oddsidemargin}{-.5in}%
\addtolength{\evensidemargin}{-.5in}%
\addtolength{\textwidth}{1in}%
\addtolength{\textheight}{1.3in}%
\addtolength{\topmargin}{-.8in}%


\begin{document}

%\bibliographystyle{natbib}

\def\spacingset#1{\renewcommand{\baselinestretch}%
{#1}\small\normalsize} \spacingset{1}


%%%%%%%%%%%%%%%%%%%%%%%%%%%%%%%%%%%%%%%%%%%%%%%%%%%%%%%%%%%%%%%%%%%%%%%%%%%%%%

\if0\blind
{
  \title{\bf Title}
  \author{Author 1\thanks{
    The authors gratefully acknowledge \textit{please remember to list all relevant funding sources in the unblinded version}}\hspace{.2cm}\\
    Department of YYY, University of XXX\\
    and \\
    Author 2 \\
    Department of ZZZ, University of WWW}
  \maketitle
} \fi

\if1\blind
{
  \bigskip
  \bigskip
  \bigskip
  \begin{center}
    {\LARGE\bf Title}
\end{center}
  \medskip
} \fi

\bigskip
\begin{abstract}
FROM JSM::: The growing prominence of data science indicates its key role in addressing social discrimination, emphasizing the need for a nuanced understanding and mitigation strategies of such biases. "Data Science Looks At Discrimination" (DSLD) is an R package designed to offer users a comprehensive set of statistical and graphical tools to assess discrimination associated with protected groups (race, gender, age, etc.) The package addresses two main issues: 1. Finding and eliminating the impact of confounders. 2. Reducing bias toward protected groups in prediction algorithms. For instance, users may conduct discrimination analysis through statistical inferences on differences in levels of the sensitive variable S via linear, logistic, and even machine learning models. In predictive scenarios, the package facilitates the mitigation of the impact of [S] by limiting the use of proxy variables [O]. The package is broadly aimed at users ranging from instructors of statistics classes to legal professionals, offering a powerful yet intuitive approach to discrimination analysis. DSLD also includes an 80-page Quarto book to serve as a guide to key statistical principles and their applications.
\end{abstract}

\noindent%
{\it Keywords:} Data Science; Fair Machine Learning; Education; Discrimination Analysis; Confounders; Statistical Analysis; Quarto Notebook
\vfill

\newpage
\spacingset{1.45} % DON'T change the spacing!
\section{Introduction}
\label{sec:intro}

\section{Detection of Discrimination}
\label{sec:detect}

\subsection{Graphics}
\subsection{Analytics}



\section{Reduction of Bias in Machine Learning}
\label{sec:ml}

\subsection{Graphics}
\subsection{Analytics}


\section{Discussion}
\label{sec:conc}


\bigskip
\begin{center}
{\large\bf SUPPLEMENTARY MATERIAL}
\end{center}

\begin{description}

\item[Title:] Brief description. (file type)

\item[R-package for  MYNEW routine:] R-package ?MYNEW? containing code to perform the diagnostic methods described in the article. The package also contains all datasets used as examples in the article. (GNU zipped tar file)

\item[HIV data set:] Data set used in the illustration of MYNEW method in Section~ 3.2. (.txt file)

\end{description}

\section{BibTeX}

We hope you've chosen to use BibTeX!\ If you have, please feel free to use the package natbib with any bibliography style you're comfortable with. The .bst file Chicago was used here, and agsm.bst has been included here for your convenience. 

\bibliographystyle{Chicago}
\bibliography{references}

\end{document}
